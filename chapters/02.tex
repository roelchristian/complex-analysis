\chapter{Complex integration}

\section{Contour integrals}

\begin{definition}
    Let \(f: [a, b] \subseteq \R \to \C\) be a complex-valued function of a real variable \(t\). Then the definite integral of \(f\) over \([a, b]\) is defined as
    \[
        \int_a^b f(t) \, dt = \int_a^b u(t) \, dt + i \int_a^b v(t) \, dt,
    \]
    where \(f(t) = u(t) + iv(t)\) with \(u\) and \(v\) continuous for all \(t \in [a, b]\).
    \label{def:complex-integral-real-variable}
\end{definition}

It follows immediately from Definition~\ref{def:complex-integral-real-variable} that
\[
    \Re \int_a^b f(t) \, dt = \int_a^b u(t) \, dt = \int_a^b \Re f(t) \, dt
\]
and
\[
    \Im \int_a^b f(t) \, dt = \int_a^b v(t) \, dt = \int_a^b \Im f(t) \, dt.
\]

\begin{example}
    Evaluate the integral
    \[
        \int_0^1 (t-i)^2 \, dt.
    \]
\end{example}

We quickly validate some rules of integration from calculus.

\begin{theorem}
    Let \(f\) and \(g\) be complex-valued functions of a real variable \(t\) and let \(z_0\) be a complex constant. Then
    \begin{enumerate}[label=(\alph*)]
        \item \(\displaystyle \int_a^b z_0 f(t) \, dt = z_0 \int_a^b f(t) \, dt\),
        \item \(\displaystyle \int_a^b \left( f(t) + g(t) \right) \, dt = \int_a^b f(t) \, dt + \int_a^b g(t) \, dt\),
        \item \(\displaystyle \int_a^b f(t) \, dt = -\int_b^a f(t) \, dt\).
        \item \(\displaystyle \int_a^b f(t) \, dt = \int_a^c f(t) \, dt + \int_c^b f(t) \, dt\).
    \end{enumerate}
\end{theorem}

We also restate the fundamental theorem of calculus.

\begin{theorem}[Fundamental theorem of calculus]
    Let \(f\) be a continuous function on \([a, b]\) and let \(F\) be a function such that \(F'(t) = f(t)\) for all \(t \in [a, b]\). Then
    \[
        \int_a^b f(t) \, dt = F(b) - F(a).
    \]
\end{theorem}

Some important results from real-variable calculus, however, do not carry over to complex-valued functions of a real variable. As we shall see in the following two examples.

\begin{example}
    Show that the mean-value theorem for derivatives does not generally hold for complex-valued functions of a real variable.

    \begin{solution}
        
    \end{solution}
\end{example}

\begin{example}
    Show that the mean-value theorem for integrals does not generally hold for functions of the form \(f(t) = u(t) + iv(t)\).

    \begin{solution}
        Consider the function \(f(t) = e^{it}\) on the interval \([0, 2\pi]\). Then
        \[
            \int_0^{2\pi} e^{it} \, dt = \left. \frac{e^{it}}{i} \right|_0^{2\pi} = 0.
        \]

        Now recall that the mean-value theorem for integrals states that if \(f\) is continuous and integrable on \([a, b]\), then there exists a \(c \in [a, b]\) such that
        \[
            f(c) = \frac{1}{b-a} \int_a^b f(t) \, dt
        \]
        so that,
        \[
            \int_a^b f(t) \, dt = f(c)(b-a).
        \]

        However for any \(c \in [0, 2\pi]\), we have
        \[
            |e^{ic}(2\pi - 0) | = 2\pi \neq 0
        \]
        and thus \(f(c)(b - a) \neq \int_a^b f(t) \, dt\) and the mean-value theorem for integrals does not hold for complex-valued functions of a real variable.
    \end{solution}
\end{example}


\begin{theorem}
    Let \(f\) be as in Definition~\ref{def:complex-integral-real-variable}. Then
    \[
        \left| \int_a^b f(t) \, dt \right| \leq \int_a^b |f(t)| \, dt.
    \]
    \label{thm:modulus-integral-inequality}
\end{theorem}

\begin{proof}
    Let the integral denote the complex number \(z = \rho e^{i\theta}\). Then
    \[
        \left| \int_a^b f(t) \, dt \right| = \left| \rho e^{i\theta} \right| = \rho.
    \]
    On the other hand, dividing \(z\) by \(e^{i\theta}\) gives
    \[
        \rho = e^{-i\theta} \int_a^b f(t) \, dt = \int_a^b e^{-i\theta} f(t) \, dt,
    \]
    which, because \(f\) is complex-valued, we can write as
    \[
        \rho = \Re \int_a^b e^{-i\theta} f(t) \, dt + i \Im \int_a^b e^{-i\theta} f(t) \, dt.
    \]
    Because \(\rho\) is real, the imaginary part of the right-hand side must vanish, so that
    \[
        \rho = \Re \int_a^b e^{-i\theta} f(t) \, dt = \int_a^b \Re e^{-i\theta} f(t) \, dt.
    \]
    Now since \(|e^{-i\theta}| = 1\) and for any complex number \(\zeta\), \(\Re \zeta \leq |\zeta|\), we have
    \[
        \rho = \int_a^b \Re e^{-i\theta} f(t) \, dt \leq \int_a^b |e^{-i\theta} f(t)| \, dt = \int_a^b |f(t)| \, dt.
    \]
\end{proof}



\begin{definition}
    Let \(\gamma\) be a contour with the equation \(z = z(t)\) for \(a \leq t \leq b\) contained in some region \(\Omega\) and let \(f\) be a function defined and continuous on \(\gamma\). Then the integral of \(f\) along \(\gamma\) is defined as
    \[
        \int_\gamma f(z) \, dz = \int_a^b f(z(t)) z'(t) \, dt.
    \]
\end{definition}

\begin{theorem}[Invariance under reparametrization]
    Let \(\gamma\) be a contour with the equation \(z = z(t)\) for \(a \leq t \leq b\) and let \(\phi: [\alpha, \beta] \to [a, b]\) be a reparametrization of \(\gamma\). Then
    \[
        \int_\gamma f(z) \, dz = \int_{\phi(\gamma)} f(z) \, dz.
    \]
\end{theorem}

\begin{proof}
    For all \(\tau \in [\alpha, \beta]\), we can find a unique \(t \in [a, b]\) such that \(t = \phi(\tau)\). Then if \(\zeta(\tau)\) is another parametrization of \(\gamma\), we have
    \[
        \zeta(\tau) = z(\phi(\tau)) = z(t).
    \]
    Differentiating with respect to \(\tau\) gives
    \[
        \zeta'(\tau) = \left( z(\phi(\tau)) \right)' = z'(\phi(\tau)) \phi'(\tau).
    \]
    Now from the definition of the integral along \(\gamma\), we have
    \begin{align*}
        \int_{\phi(\gamma)} f(z) \, dz &= \int_{\phi(\alpha)}^{\phi(\beta)} f(\zeta(\tau)) \zeta'(\tau) \, d\tau \\
        &= \int_{\phi(\alpha)}^{\phi(\beta)} f(z(\phi(\tau))) z'(\phi(\tau)) \phi'(\tau) \, d\tau \\
        &= \int_{a}^{b} f(z(t)) z'(t) \, dt\\
        &= \int_{\gamma} f(z) \, dz.
    \end{align*}
\end{proof}

\begin{theorem}
    If \(\gamma\) is a contour such that \(\gamma = \gamma_1 + \gamma_2 + \cdots + \gamma_n\), where each \(\gamma_k\) is a contour with equation \(z = z_k(t)\) for \(a_k \leq t \leq b_k\) and \(\gamma_k \cap \gamma_{k+1} = \{z_k(b_k) = z_{k+1}(a_{k+1})\}\), i.e., the contours are joined end-to-end, then
    \[
        \int_\gamma f(z) \, dz = \int_{\gamma_1} f(z) \, dz + \int_{\gamma_2} f(z) \, dz + \cdots + \int_{\gamma_n} f(z) \, dz.
    \]
\end{theorem}

\begin{example}
    Evaluate the integral
    \[
        \int_{\gamma} \conj{z} \, dz,
    \]
    where \(\gamma\) is the right-hand half of the circle of radius \(2\) centered at the origin.
\end{example}